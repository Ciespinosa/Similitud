\documentclass[]{book}
\usepackage{lmodern}
\usepackage{amssymb,amsmath}
\usepackage{ifxetex,ifluatex}
\usepackage{fixltx2e} % provides \textsubscript
\ifnum 0\ifxetex 1\fi\ifluatex 1\fi=0 % if pdftex
  \usepackage[T1]{fontenc}
  \usepackage[utf8]{inputenc}
\else % if luatex or xelatex
  \ifxetex
    \usepackage{mathspec}
  \else
    \usepackage{fontspec}
  \fi
  \defaultfontfeatures{Ligatures=TeX,Scale=MatchLowercase}
\fi
% use upquote if available, for straight quotes in verbatim environments
\IfFileExists{upquote.sty}{\usepackage{upquote}}{}
% use microtype if available
\IfFileExists{microtype.sty}{%
\usepackage{microtype}
\UseMicrotypeSet[protrusion]{basicmath} % disable protrusion for tt fonts
}{}
\usepackage[margin=1in]{geometry}
\usepackage{hyperref}
\hypersetup{unicode=true,
            pdftitle={Similitud de Comunidades biológicas},
            pdfauthor={Carlos Iván Espinosa},
            pdfborder={0 0 0},
            breaklinks=true}
\urlstyle{same}  % don't use monospace font for urls
\usepackage{natbib}
\bibliographystyle{apalike}
\usepackage{longtable,booktabs}
\usepackage{graphicx,grffile}
\makeatletter
\def\maxwidth{\ifdim\Gin@nat@width>\linewidth\linewidth\else\Gin@nat@width\fi}
\def\maxheight{\ifdim\Gin@nat@height>\textheight\textheight\else\Gin@nat@height\fi}
\makeatother
% Scale images if necessary, so that they will not overflow the page
% margins by default, and it is still possible to overwrite the defaults
% using explicit options in \includegraphics[width, height, ...]{}
\setkeys{Gin}{width=\maxwidth,height=\maxheight,keepaspectratio}
\IfFileExists{parskip.sty}{%
\usepackage{parskip}
}{% else
\setlength{\parindent}{0pt}
\setlength{\parskip}{6pt plus 2pt minus 1pt}
}
\setlength{\emergencystretch}{3em}  % prevent overfull lines
\providecommand{\tightlist}{%
  \setlength{\itemsep}{0pt}\setlength{\parskip}{0pt}}
\setcounter{secnumdepth}{5}
% Redefines (sub)paragraphs to behave more like sections
\ifx\paragraph\undefined\else
\let\oldparagraph\paragraph
\renewcommand{\paragraph}[1]{\oldparagraph{#1}\mbox{}}
\fi
\ifx\subparagraph\undefined\else
\let\oldsubparagraph\subparagraph
\renewcommand{\subparagraph}[1]{\oldsubparagraph{#1}\mbox{}}
\fi

%%% Use protect on footnotes to avoid problems with footnotes in titles
\let\rmarkdownfootnote\footnote%
\def\footnote{\protect\rmarkdownfootnote}

%%% Change title format to be more compact
\usepackage{titling}

% Create subtitle command for use in maketitle
\newcommand{\subtitle}[1]{
  \posttitle{
    \begin{center}\large#1\end{center}
    }
}

\setlength{\droptitle}{-2em}
  \title{Similitud de Comunidades biológicas}
  \pretitle{\vspace{\droptitle}\centering\huge}
  \posttitle{\par}
  \author{Carlos Iván Espinosa}
  \preauthor{\centering\large\emph}
  \postauthor{\par}
  \predate{\centering\large\emph}
  \postdate{\par}
  \date{Octubre 2016}

\usepackage{booktabs}

\begin{document}
\maketitle

{
\setcounter{tocdepth}{1}
\tableofcontents
}
\chapter*{Prefacio}\label{prefacio}
\addcontentsline{toc}{chapter}{Prefacio}

Placeholder

\chapter{Introducción}\label{introduccion}

Placeholder

\chapter{Ejercicio práctico}\label{ejercicio-practico}

\begin{center}\rule{0.5\linewidth}{\linethickness}\end{center}

\chapter{Enunciado}\label{enunciado}

Una de las preguntas básicas de un ecólogo es saber ¿Cómo de diferentes
son dos comunidades?. Como hemos visto en el capítulo anterior existen
varias decisiones que los investigadores debemos tomar, estas decisiones
afectan directamente a los resultados que podemos obtener y por ende a
las conclusiones biológicas que obtenemos de este análisis.

El presente ejercicio evaluaremos como las diferentes desiciones que
tomamos entorno al procesamiento de datos afectan nuestras medidas de
similitud, y cuales son las conclusiones biológicas que obtenemos con
uno u otro procedimiento. En la tabla \ref{tab:ejer1} mostramos cinco
comunidades hipotéticas.

\begin{table}

\caption{\label{tab:ejer1}Comunidades hipotéticas}
\centering
\begin{tabular}[t]{lrrrrrrrr}
\toprule
  & sp1 & sp2 & sp3 & sp4 & sp5 & sp6 & sp7 & sp8\\
\midrule
A & 26 & 17 & 16 & 1995 & 159 & 0 & 362 & 0\\
B & 0 & 35 & 14 & 236 & 54 & 0 & 496 & 57\\
C & 24 & 0 & 26 & 17 & 88 & 18 & 907 & 20\\
D & 35 & 18 & 24 & 2033 & 175 & 15 & 376 & 16\\
E & 105 & 129 & 40 & 18 & 191 & 53 & 964 & 134\\
\bottomrule
\end{tabular}
\end{table}

Con los datos anteriores:

\begin{enumerate}
\def\labelenumi{\alph{enumi}.}
\item
  Convierta los datos en abundancia relativa por especie (la suma en
  cada especie debe ser igual a 1). Dibuje dos gráficas para
  representar; i) la abundancia total y ii) abundancia relativa de cada
  localidad. ¿Qué diferencias puede ver en la gráfica i y en la ii?¿Qué
  implicaciones biológicas podría tener si utilizamos la primera o la
  segunda matriz para calcular las similitudes?
\item
  Calcule la distancia Euclideana y de Bray Curtis para cada sitio con
  las dos medidas de abundancia y grafíquelas utilizando el NMDS. ¿Cómo
  cambia entre distancias y abundancias? ¿Por qué se dan estas
  diferencias? ¿Puede darle una explicación biológica a los diferentes
  resultados?
\item
  Evalúe la similitud (Sorensen) y el porcentaje de similitud entre
  pares de sitios. ¿Cuáles son los sitios más similares? ¿Cuál es la
  razón de las diferencias entre los índices utilizados? ¿De una
  interpretación biológica a estos resultados?
\end{enumerate}

\chapter{Metodología}\label{metodologia}

Procese los datos como se explica en el enunciado y conteste las
preguntas. A partir de estas preguntas prepare un reporte técnico en el
cual usted explique las diferencias que existen entre las diferentes
respuestas que tuvo y sus implicaciones biológicas.

\bibliography{packages.bib}


\end{document}
