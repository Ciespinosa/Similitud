\documentclass[]{book}
\usepackage{lmodern}
\usepackage{amssymb,amsmath}
\usepackage{ifxetex,ifluatex}
\usepackage{fixltx2e} % provides \textsubscript
\ifnum 0\ifxetex 1\fi\ifluatex 1\fi=0 % if pdftex
  \usepackage[T1]{fontenc}
  \usepackage[utf8]{inputenc}
\else % if luatex or xelatex
  \ifxetex
    \usepackage{mathspec}
  \else
    \usepackage{fontspec}
  \fi
  \defaultfontfeatures{Ligatures=TeX,Scale=MatchLowercase}
\fi
% use upquote if available, for straight quotes in verbatim environments
\IfFileExists{upquote.sty}{\usepackage{upquote}}{}
% use microtype if available
\IfFileExists{microtype.sty}{%
\usepackage{microtype}
\UseMicrotypeSet[protrusion]{basicmath} % disable protrusion for tt fonts
}{}
\usepackage[margin=1in]{geometry}
\usepackage{hyperref}
\hypersetup{unicode=true,
            pdftitle={Similitud de Comunidades biológicas},
            pdfauthor={Carlos Iván Espinosa},
            pdfborder={0 0 0},
            breaklinks=true}
\urlstyle{same}  % don't use monospace font for urls
\usepackage{natbib}
\bibliographystyle{apalike}
\usepackage{color}
\usepackage{fancyvrb}
\newcommand{\VerbBar}{|}
\newcommand{\VERB}{\Verb[commandchars=\\\{\}]}
\DefineVerbatimEnvironment{Highlighting}{Verbatim}{commandchars=\\\{\}}
% Add ',fontsize=\small' for more characters per line
\usepackage{framed}
\definecolor{shadecolor}{RGB}{248,248,248}
\newenvironment{Shaded}{\begin{snugshade}}{\end{snugshade}}
\newcommand{\KeywordTok}[1]{\textcolor[rgb]{0.13,0.29,0.53}{\textbf{{#1}}}}
\newcommand{\DataTypeTok}[1]{\textcolor[rgb]{0.13,0.29,0.53}{{#1}}}
\newcommand{\DecValTok}[1]{\textcolor[rgb]{0.00,0.00,0.81}{{#1}}}
\newcommand{\BaseNTok}[1]{\textcolor[rgb]{0.00,0.00,0.81}{{#1}}}
\newcommand{\FloatTok}[1]{\textcolor[rgb]{0.00,0.00,0.81}{{#1}}}
\newcommand{\ConstantTok}[1]{\textcolor[rgb]{0.00,0.00,0.00}{{#1}}}
\newcommand{\CharTok}[1]{\textcolor[rgb]{0.31,0.60,0.02}{{#1}}}
\newcommand{\SpecialCharTok}[1]{\textcolor[rgb]{0.00,0.00,0.00}{{#1}}}
\newcommand{\StringTok}[1]{\textcolor[rgb]{0.31,0.60,0.02}{{#1}}}
\newcommand{\VerbatimStringTok}[1]{\textcolor[rgb]{0.31,0.60,0.02}{{#1}}}
\newcommand{\SpecialStringTok}[1]{\textcolor[rgb]{0.31,0.60,0.02}{{#1}}}
\newcommand{\ImportTok}[1]{{#1}}
\newcommand{\CommentTok}[1]{\textcolor[rgb]{0.56,0.35,0.01}{\textit{{#1}}}}
\newcommand{\DocumentationTok}[1]{\textcolor[rgb]{0.56,0.35,0.01}{\textbf{\textit{{#1}}}}}
\newcommand{\AnnotationTok}[1]{\textcolor[rgb]{0.56,0.35,0.01}{\textbf{\textit{{#1}}}}}
\newcommand{\CommentVarTok}[1]{\textcolor[rgb]{0.56,0.35,0.01}{\textbf{\textit{{#1}}}}}
\newcommand{\OtherTok}[1]{\textcolor[rgb]{0.56,0.35,0.01}{{#1}}}
\newcommand{\FunctionTok}[1]{\textcolor[rgb]{0.00,0.00,0.00}{{#1}}}
\newcommand{\VariableTok}[1]{\textcolor[rgb]{0.00,0.00,0.00}{{#1}}}
\newcommand{\ControlFlowTok}[1]{\textcolor[rgb]{0.13,0.29,0.53}{\textbf{{#1}}}}
\newcommand{\OperatorTok}[1]{\textcolor[rgb]{0.81,0.36,0.00}{\textbf{{#1}}}}
\newcommand{\BuiltInTok}[1]{{#1}}
\newcommand{\ExtensionTok}[1]{{#1}}
\newcommand{\PreprocessorTok}[1]{\textcolor[rgb]{0.56,0.35,0.01}{\textit{{#1}}}}
\newcommand{\AttributeTok}[1]{\textcolor[rgb]{0.77,0.63,0.00}{{#1}}}
\newcommand{\RegionMarkerTok}[1]{{#1}}
\newcommand{\InformationTok}[1]{\textcolor[rgb]{0.56,0.35,0.01}{\textbf{\textit{{#1}}}}}
\newcommand{\WarningTok}[1]{\textcolor[rgb]{0.56,0.35,0.01}{\textbf{\textit{{#1}}}}}
\newcommand{\AlertTok}[1]{\textcolor[rgb]{0.94,0.16,0.16}{{#1}}}
\newcommand{\ErrorTok}[1]{\textcolor[rgb]{0.64,0.00,0.00}{\textbf{{#1}}}}
\newcommand{\NormalTok}[1]{{#1}}
\usepackage{longtable,booktabs}
\usepackage{graphicx,grffile}
\makeatletter
\def\maxwidth{\ifdim\Gin@nat@width>\linewidth\linewidth\else\Gin@nat@width\fi}
\def\maxheight{\ifdim\Gin@nat@height>\textheight\textheight\else\Gin@nat@height\fi}
\makeatother
% Scale images if necessary, so that they will not overflow the page
% margins by default, and it is still possible to overwrite the defaults
% using explicit options in \includegraphics[width, height, ...]{}
\setkeys{Gin}{width=\maxwidth,height=\maxheight,keepaspectratio}
\IfFileExists{parskip.sty}{%
\usepackage{parskip}
}{% else
\setlength{\parindent}{0pt}
\setlength{\parskip}{6pt plus 2pt minus 1pt}
}
\setlength{\emergencystretch}{3em}  % prevent overfull lines
\providecommand{\tightlist}{%
  \setlength{\itemsep}{0pt}\setlength{\parskip}{0pt}}
\setcounter{secnumdepth}{5}
% Redefines (sub)paragraphs to behave more like sections
\ifx\paragraph\undefined\else
\let\oldparagraph\paragraph
\renewcommand{\paragraph}[1]{\oldparagraph{#1}\mbox{}}
\fi
\ifx\subparagraph\undefined\else
\let\oldsubparagraph\subparagraph
\renewcommand{\subparagraph}[1]{\oldsubparagraph{#1}\mbox{}}
\fi

%%% Use protect on footnotes to avoid problems with footnotes in titles
\let\rmarkdownfootnote\footnote%
\def\footnote{\protect\rmarkdownfootnote}

%%% Change title format to be more compact
\usepackage{titling}

% Create subtitle command for use in maketitle
\newcommand{\subtitle}[1]{
  \posttitle{
    \begin{center}\large#1\end{center}
    }
}

\setlength{\droptitle}{-2em}

  \title{Similitud de Comunidades biológicas}
    \pretitle{\vspace{\droptitle}\centering\huge}
  \posttitle{\par}
    \author{Carlos Iván Espinosa}
    \preauthor{\centering\large\emph}
  \postauthor{\par}
      \predate{\centering\large\emph}
  \postdate{\par}
    \date{Octubre 2016}

\usepackage{booktabs}

\begin{document}
\maketitle

{
\setcounter{tocdepth}{1}
\tableofcontents
}
\chapter*{Prefacio}\label{prefacio}
\addcontentsline{toc}{chapter}{Prefacio}

\begin{center}\rule{0.5\linewidth}{\linethickness}\end{center}

La comunidad biológica se refiere a una agrupación de poblaciones de
especies que se presentan juntas en el espacio y el tiempo (Begon et al.
1999). Este concepto plantea que las comunidades tienen unos límites en
el espacio y el tiempo, y que estos límites están dados por la
distribución de las poblaciones. Sin embargo, la distribución de las
poblaciones no es homogénea y cada población responde diferente en el
espacio y el tiempo. Estas respuestas diferenciadas de cada población y
de la comunidad en general genera la zonación y la sucesión
respectivamente.

La identificación de los las diferentes formaciones biológicas en el
espacio (Zonación) o definir las etapas seriales a lo largo del tiempo
(sucesión) implica que tenemos la capacidad de establecer en que momento
una comunidad cambia. Parece una tarea sencilla, pero realmente no lo
es, ¿cuanto deberia cambiar una comunidad para poder hablar de etapas
seriales o zonas distintas? y ¿cómo podemos calcular ese cambio? Una de
las formas de responder estas preguntas puede ser intentar cuantificar
las similitudes entre localidades.

\chapter*{Objetivos}\label{objetivos}
\addcontentsline{toc}{chapter}{Objetivos}

\begin{center}\rule{0.5\linewidth}{\linethickness}\end{center}

En este ejercicio mostramos las bases del calculo cálculo de similitudes
entre comunidades, el cual se convierte en la base de los análisis
multivariantes de la comunidad. Específicamente nos interesa;

\begin{itemize}
\tightlist
\item
  Comprender las bases teóricas para el cálculo de similitudes de la
  estructura de la comunidad entre localidades.
\item
  Utilizar herramientas de análisis para calcular índices de similitud y
  distancias entre comunidades
\end{itemize}

\begin{figure}[htbp]
\centering
\includegraphics{lagar.jpg}
\caption{\emph{Stenocercus iridicens}}
\end{figure}

\chapter{Introducción}\label{introduccion}

\begin{center}\rule{0.5\linewidth}{\linethickness}\end{center}

\section{Similitud basada en la
distancia}\label{similitud-basada-en-la-distancia}

\begin{quote}
``La abundancia se refiere al número de individuos de una especie en una
determinada área''

--- (Smith and Smith 2010)
\end{quote}

Cuando hablamos de la composición de especies de una comunidad nos
referimos al conjunto de especies que habitan una determinada localidad.
Típicamente, esto incluye cierto grado de abundancia de cada especie,
pero puede también ser simplemente un listado de especies en esa
localidad, donde se registra la presencia o ausencia de cada especie.
Ahora, imaginemos que tenemos cuatro localidades (A, B, C, D) donde
recogemos los datos de densidad de dos especies; \emph{Tabebuia
billbergii} y \emph{Geofroea spinosa}, especies características de
bosques secos tropicales. Podemos introducir datos hipotéticos de
abundancia para cada especie en cada una de las localidades.

\begin{Shaded}
\begin{Highlighting}[]
\NormalTok{dens <-}\StringTok{ }\KeywordTok{data.frame}\NormalTok{(}\DataTypeTok{T.bil =} \KeywordTok{c}\NormalTok{(}\DecValTok{1}\NormalTok{, }\DecValTok{1}\NormalTok{, }\DecValTok{2}\NormalTok{, }\DecValTok{3}\NormalTok{), }\DataTypeTok{G.spi =} \KeywordTok{c}\NormalTok{(}\DecValTok{21}\NormalTok{, }\DecValTok{8}\NormalTok{, }\DecValTok{13}\NormalTok{, }\DecValTok{5}\NormalTok{)) }
\KeywordTok{row.names}\NormalTok{(dens) <-}\StringTok{ }\NormalTok{LETTERS[}\DecValTok{1}\NormalTok{:}\DecValTok{4}\NormalTok{]}
\NormalTok{dens}
\end{Highlighting}
\end{Shaded}

\begin{verbatim}
##   T.bil G.spi
## A     1    21
## B     1     8
## C     2    13
## D     3     5
\end{verbatim}

Generamos un gráfico para ver cuánto se parece cada sitio (Figura
\ref{fig:NMDS})

\begin{Shaded}
\begin{Highlighting}[]
\KeywordTok{par}\NormalTok{(}\DataTypeTok{mar=}\KeywordTok{c}\NormalTok{(}\DecValTok{4}\NormalTok{,}\DecValTok{4}\NormalTok{,}\DecValTok{1}\NormalTok{,}\DecValTok{1}\NormalTok{), }\DataTypeTok{mgp=}\KeywordTok{c}\NormalTok{(}\DecValTok{1}\NormalTok{,}\FloatTok{0.3}\NormalTok{,}\DecValTok{0}\NormalTok{), }\DataTypeTok{tcl=} \NormalTok{-}\FloatTok{0.2}\NormalTok{)}
\KeywordTok{plot}\NormalTok{(dens, }\DataTypeTok{type =} \StringTok{"n"}\NormalTok{, }\DataTypeTok{cex.axis=}\FloatTok{0.8}\NormalTok{, }\DataTypeTok{xlim=}\KeywordTok{c}\NormalTok{(}\DecValTok{0}\NormalTok{,}\DecValTok{20}\NormalTok{)) }
\KeywordTok{text}\NormalTok{(dens, }\KeywordTok{row.names}\NormalTok{(dens), }\DataTypeTok{col =}\StringTok{"blue"}\NormalTok{)}
\end{Highlighting}
\end{Shaded}

\begin{figure}[htbp]
\centering
\includegraphics{Similitud_files/figure-latex/NMDS-1.pdf}
\caption{\label{fig:NMDS}Distancias de cuatro localidades hipotéticas}
\end{figure}

En la figura \ref{fig:NMDS} vemos que la composición de especies en el
sitio A es diferente de la composición del sitio D. Es decir, la
distancia entre el sitio A y D es mayor que entre los otros sitios. Lo
siguiente que nos deberíamos preguntar es; ¿qué tan distantes están los
dos sitios? Claramente, esto depende de la escala de medición (los
valores de los ejes), y sobre cómo medimos la distancia a través del
espacio multivariado \citep{Stevens2009}.

Estas diferencias entre sitios son dependientes de la abundancia de cada
especie. En el caso de \emph{G. spinosa} su eje varía entre 5 y 21,
mientras que para \emph{T. billbergii} varía entre 1 y 3. Veamos ahora
que sucede con las similitud si incremento la abundancia de \emph{T.
billbergii}.

\begin{Shaded}
\begin{Highlighting}[]
\KeywordTok{par}\NormalTok{(}\DataTypeTok{mar=}\KeywordTok{c}\NormalTok{(}\DecValTok{4}\NormalTok{,}\DecValTok{4}\NormalTok{,}\DecValTok{1}\NormalTok{,}\DecValTok{1}\NormalTok{), }\DataTypeTok{mgp=}\KeywordTok{c}\NormalTok{(}\DecValTok{1}\NormalTok{,}\FloatTok{0.3}\NormalTok{,}\DecValTok{0}\NormalTok{), }\DataTypeTok{tcl=} \NormalTok{-}\FloatTok{0.2}\NormalTok{)}
\NormalTok{dens1 <-}\StringTok{ }\NormalTok{dens}
\NormalTok{dens1$T.bil <-}\StringTok{ }\NormalTok{dens1$T.bil*}\DecValTok{100}
\KeywordTok{plot}\NormalTok{(dens1, }\DataTypeTok{type =} \StringTok{"n"}\NormalTok{, }\DataTypeTok{cex.axis=}\FloatTok{0.8}\NormalTok{, }\DataTypeTok{ylim=}\KeywordTok{c}\NormalTok{(}\DecValTok{0}\NormalTok{,}\DecValTok{300}\NormalTok{)) }
\KeywordTok{text}\NormalTok{(dens1, }\KeywordTok{row.names}\NormalTok{(dens1), }\DataTypeTok{col =}\StringTok{"blue"}\NormalTok{)}
\end{Highlighting}
\end{Shaded}

\begin{figure}[htbp]
\centering
\includegraphics{Similitud_files/figure-latex/NMDS2-1.pdf}
\caption{\label{fig:NMDS2}Distancias de cuatro localidades hipotéticas}
\end{figure}

Como vemos en la figura \ref{fig:NMDS2} las distancias entre cada uno de
los sitios cambio, aunque la comunidad se mantuvo igual. Una forma de
corregir esta distorsión es calcular la densidad relativa de cada
especie, de esta forma cada especie variará entre 0 y 1
\citep{Stevens2009}. Cuando nos referimos a densidad relativa hablamos
de la densidad de una especie con referencia a algo, en relación a la
abundancia de individuos de la misma especie en otros sitios.

Para calcular la densidad relativa dividimos la abundancia de cada
especie para la suma total de los individuos de las especies en esa
muestra.

\begin{Shaded}
\begin{Highlighting}[]
\NormalTok{dens[,}\DecValTok{1}\NormalTok{]/}\KeywordTok{sum}\NormalTok{(dens[,}\DecValTok{1}\NormalTok{])}
\end{Highlighting}
\end{Shaded}

\begin{verbatim}
## [1] 0.1428571 0.1428571 0.2857143 0.4285714
\end{verbatim}

\begin{Shaded}
\begin{Highlighting}[]
\NormalTok{dens1[,}\DecValTok{1}\NormalTok{]/}\KeywordTok{sum}\NormalTok{(dens1[,}\DecValTok{1}\NormalTok{])}
\end{Highlighting}
\end{Shaded}

\begin{verbatim}
## [1] 0.1428571 0.1428571 0.2857143 0.4285714
\end{verbatim}

Ahora podemos ver cómo \emph{T. billbergii} varía en su abundancia en
los cuatro sitios. El sitio A y B tienen el 14\% de individuos mientras
que el D tiene el 42\% de los individuos de esta especie.
Interesantemente, no hay diferencias en las proporciones entre las dos
medidas que tenemos. ¿Qué pasó con las distancias?

\begin{Shaded}
\begin{Highlighting}[]
\NormalTok{dens3 <-}\StringTok{ }\NormalTok{dens}

\NormalTok{for(i in }\DecValTok{1}\NormalTok{:}\DecValTok{2}\NormalTok{)\{}
\NormalTok{dens3[,i] <-}\StringTok{ }\NormalTok{dens[,i]/}\KeywordTok{sum}\NormalTok{(dens[,i])}
\NormalTok{\}}
\NormalTok{dens4 <-}\StringTok{ }\NormalTok{dens1}

\NormalTok{for(i in }\DecValTok{1}\NormalTok{:}\DecValTok{2}\NormalTok{)\{}
\NormalTok{dens4[,i] <-}\StringTok{ }\NormalTok{dens1[,i]/}\KeywordTok{sum}\NormalTok{(dens1[,i])}
\NormalTok{\}}

\KeywordTok{par}\NormalTok{(}\DataTypeTok{mfcol=}\KeywordTok{c}\NormalTok{(}\DecValTok{2}\NormalTok{,}\DecValTok{2}\NormalTok{), }\DataTypeTok{mar=}\KeywordTok{c}\NormalTok{(}\DecValTok{4}\NormalTok{,}\DecValTok{4}\NormalTok{,}\DecValTok{1}\NormalTok{,}\DecValTok{1}\NormalTok{), }\DataTypeTok{mgp=}\KeywordTok{c}\NormalTok{(}\DecValTok{1}\NormalTok{,}\FloatTok{0.3}\NormalTok{,}\DecValTok{0}\NormalTok{), }\DataTypeTok{tcl=} \NormalTok{-}\FloatTok{0.2}\NormalTok{)}
\KeywordTok{plot}\NormalTok{(dens, }\DataTypeTok{type =} \StringTok{"n"}\NormalTok{, }\DataTypeTok{cex.axis=}\FloatTok{0.8}\NormalTok{, }\DataTypeTok{xlim=}\KeywordTok{c}\NormalTok{(}\DecValTok{0}\NormalTok{,}\DecValTok{20}\NormalTok{), }\DataTypeTok{main =} \StringTok{"Densidad"}\NormalTok{) }
\KeywordTok{text}\NormalTok{(dens, }\KeywordTok{row.names}\NormalTok{(dens), }\DataTypeTok{col =}\StringTok{"blue"}\NormalTok{)}

\KeywordTok{plot}\NormalTok{(dens1, }\DataTypeTok{type =} \StringTok{"n"}\NormalTok{, }\DataTypeTok{cex.axis=}\FloatTok{0.8}\NormalTok{, }\DataTypeTok{ylim=}\KeywordTok{c}\NormalTok{(}\DecValTok{0}\NormalTok{,}\DecValTok{300}\NormalTok{), }\DataTypeTok{main =} \StringTok{"Densidad 2"}\NormalTok{) }
\KeywordTok{text}\NormalTok{(dens1, }\KeywordTok{row.names}\NormalTok{(dens1), }\DataTypeTok{col =}\StringTok{"blue"}\NormalTok{)}

\KeywordTok{plot}\NormalTok{(dens3, }\DataTypeTok{type =} \StringTok{"n"}\NormalTok{, }\DataTypeTok{cex.axis=}\FloatTok{0.8}\NormalTok{, }\DataTypeTok{main =} \StringTok{"Densidad relativa"}\NormalTok{) }
\KeywordTok{text}\NormalTok{(dens3, }\KeywordTok{row.names}\NormalTok{(dens3), }\DataTypeTok{col =}\StringTok{"blue"}\NormalTok{)}

\KeywordTok{plot}\NormalTok{(dens4, }\DataTypeTok{type =} \StringTok{"n"}\NormalTok{, }\DataTypeTok{cex.axis=}\FloatTok{0.8}\NormalTok{, }\DataTypeTok{main =} \StringTok{"Densidad relativa2"}\NormalTok{) }
\KeywordTok{text}\NormalTok{(dens4, }\KeywordTok{row.names}\NormalTok{(dens4), }\DataTypeTok{col =}\StringTok{"blue"}\NormalTok{)}
\end{Highlighting}
\end{Shaded}

\begin{figure}[htbp]
\centering
\includegraphics{Similitud_files/figure-latex/NMDS3-1.pdf}
\caption{\label{fig:NMDS3}Distancias de cuatro localidades hipotéticas}
\end{figure}

En la figura \ref{fig:NMDS3} podemos apreciar que no hay diferencias
entre las dos densidades cuando estoy usando la densidad relativa. Pero
¿Qué implicaciones biológicas tiene el usar las densidades relativas
para calcular la distancia entre sitios?

Cuando usamos las densidades relativas lo que estamos haciendo es darles
el mismo peso a tods las especies, de esta manera si yo tengo un
ecosistema con una especie dominante y varias subordinadas, al usar la
densidad relativa estoy eliminando esa dominancia. Es importante que
tener claro este punto ya que las interpretaciones que puedo hacer con
los datos de densidad y densidad relativa son distintos.

Ya sea que nuestras medidas de abundancia son absoluta o relativa, nos
interesa conocer cuan diferente es la comunidad de una muestra (o sitio)
con relación a la otra. En el ejemplo ha sido fácil entender la
diferencia entre las dos comunidades debido a que teníamos únicamente
dos especies, pero con más de tres especies es complicado observar estas
diferencias gráficamente. Tal vez la forma más sencilla de describir la
diferencia entre los sitios es calcular las \emph{distancias} entre cada
par de sitios.

\subsection{Distancias entre sitios}\label{distancias-entre-sitios}

La \emph{distancia} entre dos muestras está dada por la diferencia entre
la abundancia y la composición de especies, como lo hemos visto esto
genera una distancia, en el caso del ejemplo la comunidad A esta más
alejada de la comunidad D que de las otras dos.

Existen muchas formas de poder calcular las distancias entre estos
puntos una de las más sencillas es la distancia \emph{Euclidiana}. La
distancia euclidiana entre dos sitios es simplemente la longitud del
vector que conecta los sitios y la podemos obtener como
\(\sqrt{x^2+y^2}\), donde \emph{``x''} y \emph{``y''} son las
coordenadas (x, y) de distancia entre un par de sitios.

En nuestro caso si queremos comparar B y C tenemos que la distancia en
el eje \emph{x} es la diferencia de la abundancia de \emph{T. bilbergii}
entre el sitio B y C.

\begin{Shaded}
\begin{Highlighting}[]
\NormalTok{x <-}\StringTok{ }\NormalTok{dens[}\DecValTok{2}\NormalTok{, }\DecValTok{1}\NormalTok{] -}\StringTok{ }\NormalTok{dens[}\DecValTok{3}\NormalTok{, }\DecValTok{1}\NormalTok{]}
\end{Highlighting}
\end{Shaded}

Mientras que la distancia en el eje \emph{y} es la diferencia en la
abundancia de \emph{G. spinosa} entre el sitio B y C.

\begin{Shaded}
\begin{Highlighting}[]
\NormalTok{y <-}\StringTok{ }\NormalTok{dens[}\DecValTok{2}\NormalTok{, }\DecValTok{2}\NormalTok{] -}\StringTok{ }\NormalTok{dens[}\DecValTok{3}\NormalTok{, }\DecValTok{2}\NormalTok{]}
\end{Highlighting}
\end{Shaded}

Ahora obtenemos las distancias entre los dos sitios

\begin{Shaded}
\begin{Highlighting}[]
\KeywordTok{sqrt}\NormalTok{(x^}\DecValTok{2} \NormalTok{+}\StringTok{ }\NormalTok{y^}\DecValTok{2}\NormalTok{)}
\end{Highlighting}
\end{Shaded}

\begin{verbatim}
## [1] 5.09902
\end{verbatim}

Pero como en \emph{R} todo es sencillo podemos utilizar la función
\emph{dist}

\begin{Shaded}
\begin{Highlighting}[]
\KeywordTok{dist}\NormalTok{(dens)}
\end{Highlighting}
\end{Shaded}

\begin{verbatim}
##           A         B         C
## B 13.000000                    
## C  8.062258  5.099020          
## D 16.124515  3.605551  8.062258
\end{verbatim}

Si bien este cálculo es sencillo con dos especies, si tenemos que
calcular la distancia para una comunidad con más de tres especies los
cálculos son tediosos y largos. Para calcular la distancia
\emph{Euclidiana} entre pares de sitios con \emph{R} especies utilizamos
la siguiente ecuación:

\begin{quote}
\[D_E = \sqrt{\sum_{i=l}^R (x_{ai} - x_{bi})^2}\] Distancia Euclidiana
\end{quote}

Existen otras formas de medir distancias entre dos localidades. En
ecología una de las distancias más utilizada es la distancia de
\emph{Bray-Curtis}, conocida también como \emph{Sorensen}. Esta
distancia es calculada como:

\begin{quote}
\[D_{BC} = \sum_{i=l}^R \frac{(x_{ai} - x_{bi})}{(x_{ai} + x_{bi})}\]
Distancia de Bray-Curtis
\end{quote}

La distancia \emph{Bray-Curtis} no es más que la diferencia total en la
abundancia de especies entre dos sitios, dividido para la abundancia
total en cada sitio. La distancia Bray-Curtis tiende a resultar más
intuitiva debido a que las especies comunes y raras tienen pesos
relativamente similares, mientras que la distancia euclidia depende en
mayor medida de las especies más abundantes. Esto sucede porque las
distancias euclidianas se basan en diferencias al cuadrado, mientras que
Bray-Curtis utiliza diferencias absolutas. El elevar un número al
cuadrado siempre amplifica la importancia de los valores más grandes. En
la figura \ref{fig:bray} se compara gráficos basados en distancias
euclidianas y Bray-Curtis de los mismos datos.

Como se había comentado es virtualmente imposible representar una
distancia en más de tres dimensiones (cada especie es una dimensión).
Una forma sencilla de mostrar distancias para tres o más especies es
crear un gráfico de dos dimensiones, intentando organizar todos los
sitios para que las distancias sean aproximadamente las correctas. Está
claro que esto es una aproximación nunca estas serán exactas. Una
técnica que intenta crear un arreglo aproximado es escalamiento
multidimensional no métrico (NMDS). Vamos a calcular las distancias para
nuestra comunidad, primero vamos a añadir dos especies más a nuestra
comunidad, \emph{Ceiba trichistandra} y \emph{Colicodendron scabridum}.

\begin{Shaded}
\begin{Highlighting}[]
\NormalTok{dens$C.tri<-}\StringTok{ }\KeywordTok{c}\NormalTok{(}\DecValTok{11}\NormalTok{, }\DecValTok{3}\NormalTok{, }\DecValTok{7}\NormalTok{, }\DecValTok{5}\NormalTok{)}
\NormalTok{dens$C.sca<-}\StringTok{ }\KeywordTok{c}\NormalTok{(}\DecValTok{16}\NormalTok{, }\DecValTok{0}\NormalTok{, }\DecValTok{9}\NormalTok{, }\DecValTok{4}\NormalTok{)}
\end{Highlighting}
\end{Shaded}

La función de escalamiento multidimensional no-métrico está en el
paquete \texttt{vegan}. Aquí mostramos las distancias euclidianas entre
sitios (Figura \ref{fig:bray}a) y las distancias de Bray-Curtis (Figura
\ref{fig:bray}b).

\begin{Shaded}
\begin{Highlighting}[]
\KeywordTok{library}\NormalTok{(vegan) }

\CommentTok{#Distancia Euclidiana}
\NormalTok{mdsE <-}\StringTok{ }\KeywordTok{metaMDS}\NormalTok{(dens, }\DataTypeTok{distance =} \StringTok{"euc"}\NormalTok{, }\DataTypeTok{autotransform =} \OtherTok{FALSE}\NormalTok{, }\DataTypeTok{trace =} \DecValTok{0}\NormalTok{) }
\CommentTok{#Distancia de Bray-Curtis}
\NormalTok{mdsB <-}\StringTok{ }\KeywordTok{metaMDS}\NormalTok{(dens, }\DataTypeTok{distance =} \StringTok{"bray"}\NormalTok{, }\DataTypeTok{autotransform =} \OtherTok{FALSE}\NormalTok{, }\DataTypeTok{trace =} \DecValTok{0}\NormalTok{) }
\end{Highlighting}
\end{Shaded}

\begin{Shaded}
\begin{Highlighting}[]
\KeywordTok{par}\NormalTok{(}\DataTypeTok{mfcol=}\KeywordTok{c}\NormalTok{(}\DecValTok{1}\NormalTok{,}\DecValTok{2}\NormalTok{), }\DataTypeTok{oma=}\KeywordTok{c}\NormalTok{(}\DecValTok{1}\NormalTok{,}\DecValTok{1}\NormalTok{,}\DecValTok{1}\NormalTok{,}\DecValTok{1}\NormalTok{), }\DataTypeTok{mar=}\KeywordTok{c}\NormalTok{(}\DecValTok{4}\NormalTok{,}\DecValTok{4}\NormalTok{,}\DecValTok{1}\NormalTok{,}\DecValTok{1}\NormalTok{),}
    \DataTypeTok{mgp=}\KeywordTok{c}\NormalTok{(}\DecValTok{1}\NormalTok{,}\FloatTok{0.3}\NormalTok{,}\DecValTok{0}\NormalTok{), }\DataTypeTok{tcl=} \NormalTok{-}\FloatTok{0.2}\NormalTok{)}

\KeywordTok{plot}\NormalTok{(mdsE, }\DataTypeTok{display =} \StringTok{"sites"}\NormalTok{, }
     \DataTypeTok{type =} \StringTok{"text"}\NormalTok{,}\DataTypeTok{main=}\StringTok{"a)Euclidiana"}\NormalTok{, }
     \DataTypeTok{cex.axis=} \FloatTok{0.7}\NormalTok{, }\DataTypeTok{cex.main=}\FloatTok{0.75}\NormalTok{, }\DataTypeTok{cex.lab=}\FloatTok{0.7}\NormalTok{)}

\KeywordTok{plot}\NormalTok{(mdsB, }\DataTypeTok{display =} \StringTok{"sites"}\NormalTok{, }\DataTypeTok{type =} \StringTok{"text"}\NormalTok{, }
     \DataTypeTok{main=}\StringTok{"b)Bray-Curtis"}\NormalTok{, }
     \DataTypeTok{cex.axis=} \FloatTok{0.7}\NormalTok{, }\DataTypeTok{cex.main=}\FloatTok{0.75}\NormalTok{, }\DataTypeTok{cex.lab=}\FloatTok{0.7}\NormalTok{)}
\end{Highlighting}
\end{Shaded}

\begin{figure}[htbp]
\centering
\includegraphics{Similitud_files/figure-latex/bray-1.pdf}
\caption{\label{fig:bray}Arreglo de las parcelas en distancias
multidimensionales no métricas (NMDS). Estas dos figuras muestran los
mismos datos en bruto, pero las distancias euclidianas tienden a
enfatizar las diferencias debidas a las especies más abundantes,
mientras que Bray-Curtis no lo hace.}
\end{figure}

\section{Similitud}\label{similitud}

Ahora que sabemos cuan distantes son los diferentes sitios, muchas veces
nos podría interesar cuan similares son cada uno de los sitios a
continuación se describen dos medidas de similitud; \emph{Porcentaje de
Similitud} e \emph{Índice de Sorensen}.

El \emph{porcentaje de similitud} puede ser simplemente la suma de los
porcentajes mínimos de cada especie en la comunidad. Lo primero que
debemos hacer es convertir la abundancia de cada especie a su abundancia
relativa dentro de cada sitio. Para ello dividimos la abundancia de cada
especie por la suma de las abundancias en cada sitio.

\begin{Shaded}
\begin{Highlighting}[]
\NormalTok{dens.RA <-}\StringTok{ }\KeywordTok{t}\NormalTok{(}\KeywordTok{apply}\NormalTok{(dens, }\DecValTok{1}\NormalTok{, function(sp.abun) sp.abun/}\KeywordTok{sum}\NormalTok{(sp.abun)))}
\NormalTok{dens.RA}
\end{Highlighting}
\end{Shaded}

\begin{verbatim}
##        T.bil     G.spi     C.tri     C.sca
## A 0.02040816 0.4285714 0.2244898 0.3265306
## B 0.08333333 0.6666667 0.2500000 0.0000000
## C 0.06451613 0.4193548 0.2258065 0.2903226
## D 0.17647059 0.2941176 0.2941176 0.2352941
\end{verbatim}

El siguiente paso para comparar entre sitios, es encontrar el valor
mínimo para cada especie entre los sitios que debemos comparar. Vamos a
comparar los sitios A y B, para esto utilizamos la función
\texttt{aplly}, la cual nos permite encontrar el valor mínimo entre las
filas 1 y 2 (sitio A y B respectivamente). Para \emph{T. billbergi} en
el sitio A la abundancia relativa es 0.02 que es menor a la abundancia
en el sitio B que es de 0.08.

\begin{Shaded}
\begin{Highlighting}[]
\NormalTok{mins <-}\StringTok{ }\KeywordTok{apply}\NormalTok{(dens.RA[}\DecValTok{1}\NormalTok{:}\DecValTok{2}\NormalTok{, ], }\DecValTok{2}\NormalTok{, min)}
\NormalTok{mins}
\end{Highlighting}
\end{Shaded}

\begin{verbatim}
##      T.bil      G.spi      C.tri      C.sca 
## 0.02040816 0.42857143 0.22448980 0.00000000
\end{verbatim}

Finalmente para conocer el porcentaje de similitud entre los dos sitios
sumamos estos valores y multiplicamos por 100.

\begin{Shaded}
\begin{Highlighting}[]
\KeywordTok{sum}\NormalTok{(mins)*}\DecValTok{100}
\end{Highlighting}
\end{Shaded}

\begin{verbatim}
## [1] 67.34694
\end{verbatim}

Esto significa que la comunidad A y B tienen un porcentaje de similitud
del 67\%.

El índice de Sorensen es la segunda medida de similitud que vamos a
estudiar, este índice es medido como:

\begin{quote}
\[S_s= \frac{(2C)}{(A+B)}\] Índice de Sorensen
\end{quote}

Donde \emph{C} es el número de especies en común entre los dos sitios, y
\emph{A} y \emph{B} son el número de especies en cada sitio. Esto es
equivalente a dividir las especies compartidas por la riqueza media.

Para calcular el índice de Sorensen entre los sitios A y B necesitamos
definir el número de especies compartidas y luego la riqueza de cada uno
de los dos sitios.

Definimos si alguna de las especies en uno de los sitios la abundancia
no es igual a cero, eso nos dirá en qué casos se comparten especies.
Finalmente, sumamos todas las especies que su abundancia es mayor a
cero.

\begin{Shaded}
\begin{Highlighting}[]
\NormalTok{comp<-}\StringTok{ }\KeywordTok{apply}\NormalTok{(dens[}\DecValTok{1}\NormalTok{:}\DecValTok{2}\NormalTok{, ], }\DecValTok{2}\NormalTok{, function(abuns) }\KeywordTok{all}\NormalTok{(abuns !=}\StringTok{ }\DecValTok{0}\NormalTok{))}
\NormalTok{comp}
\end{Highlighting}
\end{Shaded}

\begin{verbatim}
## T.bil G.spi C.tri C.sca 
##  TRUE  TRUE  TRUE FALSE
\end{verbatim}

\begin{Shaded}
\begin{Highlighting}[]
\NormalTok{Rs <-}\StringTok{ }\KeywordTok{apply}\NormalTok{(dens[}\DecValTok{1}\NormalTok{:}\DecValTok{2}\NormalTok{, ], }\DecValTok{1}\NormalTok{, function(x) }\KeywordTok{sum}\NormalTok{(x >}\StringTok{ }\DecValTok{0}\NormalTok{))}
\NormalTok{Rs}
\end{Highlighting}
\end{Shaded}

\begin{verbatim}
## A B 
## 4 3
\end{verbatim}

Como vemos, la abundancia de \emph{C. scabridum} en uno de los dos
sitios es igual a Cero, lo confirmamos al tener la riqueza por sitio. El
sitio B tenemos únicamente 3 especies.

Ahora aplicamos la formula, dividimos las especies compartidas
(\emph{comp}) para la riqueza total de los dos sitios y lo multiplicamos
por 2.

\begin{Shaded}
\begin{Highlighting}[]
\NormalTok{(}\DecValTok{2}\NormalTok{*}\KeywordTok{sum}\NormalTok{(comp))/}\KeywordTok{sum}\NormalTok{(Rs)}
\end{Highlighting}
\end{Shaded}

\begin{verbatim}
## [1] 0.8571429
\end{verbatim}

Según el índice de Sorensen estos dos sitios son parecidos en un 86\%.
Los datos de los dos índices utilizados difieren entre sí, el porcentaje
de similitud utiliza no solamente la presencia ausencia sino también la
abundancia lo que podría estar reduciendo la similitud entre sitios.

\begin{FOO}
La distancia de Bray Curtis y el índice de Sorensen tienen una base
conceptual similar, la diferencia es que Bray-Curtis se basa en datos de
abundancias y Sorensen en datos de presencia ausencia. Muchas veces se
utilizan como sinónimos aunque se especifíca si esta basado en
abundancias o en presencia-ausencia.
\end{FOO}

\section{Transformación y Estandarización de
datos}\label{transformacion-y-estandarizacion-de-datos}

Cuando trabajamos con datos multivariantes cabe la posibilidad de que
los datos dentro de esta matriz tengan diferencias de medidas
importantes. Como vimos antes el cálculo de distancia entre los sitios
puede verse fuertemente afectado por los datos y la magnitud de sus
diferencias.

\begin{Shaded}
\begin{Highlighting}[]
\KeywordTok{library}\NormalTok{(readxl)}
\KeywordTok{library}\NormalTok{(knitr)}
\NormalTok{dta <-}\StringTok{ }\KeywordTok{read_excel}\NormalTok{(}\StringTok{"bentos.xlsx"}\NormalTok{)}
\KeywordTok{kable}\NormalTok{(dta)}
\end{Highlighting}
\end{Shaded}

\begin{tabular}{l|r|r|r|r|r|r|r|r|r|r|r}
\hline
LOCALIDAD & Alluaudomyia & Atopsyche & Atrichopogon & Baetis & Bezzia & Blepharicera & Ceratopogonidae & Chelifera & Chimarra & Chironominae mfe1 & Colembola mf1\\
\hline
Bo-1 & 0 & 0 & 0 & 6 & 0 & 1 & 0 & 1 & 3 & 18 & 4\\
\hline
Bo-2 & 0 & 0 & 0 & 3 & 0 & 0 & 0 & 0 & 1 & 9 & 0\\
\hline
Bo-3 & 0 & 0 & 0 & 6 & 0 & 0 & 1 & 1 & 1 & 9 & 0\\
\hline
BP-1 & 0 & 3 & 0 & 81 & 0 & 0 & 0 & 0 & 0 & 27 & 0\\
\hline
BP-2 & 0 & 0 & 0 & 9 & 0 & 0 & 0 & 0 & 2 & 0 & 0\\
\hline
BP-3 & 0 & 0 & 0 & 54 & 0 & 0 & 1 & 0 & 0 & 9 & 0\\
\hline
Pa-1 & 1 & 0 & 0 & 984 & 0 & 0 & 0 & 0 & 0 & 81 & 0\\
\hline
Pa-2 & 0 & 0 & 0 & 15 & 0 & 0 & 0 & 0 & 1 & 9 & 0\\
\hline
Pa-3 & 0 & 0 & 0 & 93 & 1 & 0 & 0 & 0 & 0 & 18 & 0\\
\hline
Ur-1 & 0 & 0 & 0 & 6 & 0 & 0 & 0 & 0 & 0 & 855 & 0\\
\hline
Ur-2 & 0 & 0 & 1 & 12 & 0 & 0 & 0 & 1 & 0 & 9 & 0\\
\hline
Ur-3 & 0 & 0 & 0 & 0 & 10 & 0 & 0 & 0 & 0 & 27 & 0\\
\hline
\end{tabular}

La transformación

Imaginemos que estamos trabajando con la comunidad de herbáceas y
existan especies con abundancias máximas de 500, mientras otras especies
no alcanzan 10 individuos, o en comunidades de insectos donde ciertas
especies puedan tener valores de miles y otros de decenas. Si no
transformamos la ordenación estará determinada básicamente por esta
especie.

Al transformar los datos evitamos que las especies más comunes dominen
en el resultado final de la ordenación y aumentamos la influencia de las
especies subordinadas en el modelo resultante.

Existen varias posibilidades para transformar los datos, por lo que
definir que función utilizar para transformar los datos es importante.
Cada tipo de transformación produce resultados distintos por lo que
debemos utilizarlas con precaución.

Las transformaciones más sencilla o menos intensa es la raíz cuadrada,
mientras que el logaritmo es la transformación más intensa, podríamos
utilizar la raíz cuarta como una función intermedia. La raíz cuadrada la
utilizaríamos cuando tenemos diferencias como en el caso de las aves con
variaciones de una magnitud de diferencia (entre decenas y centenas),
mientras que la transformación logarítmica la haríamos con la comunidad
de insectos donde hay más de una magnitud de diferencia (entre decenas y
miles).

Aunque hay muchos autores que aconsejan realizar transformaciones hay
que ser conscientes de lo que estamos haciendo, transformaciones muy
fuertes en una matriz con pocas diferencias pueden hacer que, por
ejemplo, las especies raras tengan igual peso que las dominantes, esto
es lo que queremos?

Veamos un ejemplo:

\begin{Shaded}
\begin{Highlighting}[]
\KeywordTok{set.seed}\NormalTok{(}\DecValTok{4}\NormalTok{)}
\NormalTok{aves<-}\StringTok{ }\KeywordTok{data.frame}\NormalTok{(}\DataTypeTok{sp1=} \KeywordTok{sample}\NormalTok{(}\DecValTok{1}\NormalTok{:}\DecValTok{90}\NormalTok{, }\DecValTok{10}\NormalTok{), }\DataTypeTok{sp2=} \KeywordTok{sample}\NormalTok{(}\DecValTok{100}\NormalTok{:}\DecValTok{250}\NormalTok{, }\DecValTok{10}\NormalTok{))}

\NormalTok{insectos<-}\StringTok{ }\KeywordTok{data.frame}\NormalTok{(}\DataTypeTok{sp1=} \KeywordTok{sample}\NormalTok{(}\DecValTok{5}\NormalTok{:}\DecValTok{99}\NormalTok{, }\DecValTok{10}\NormalTok{), }\DataTypeTok{sp2=} \KeywordTok{sample}\NormalTok{(}\DecValTok{1000}\NormalTok{:}\DecValTok{2500}\NormalTok{, }\DecValTok{10}\NormalTok{))}

\NormalTok{##¿Qué pasa cuando transformamos?}
\KeywordTok{cbind}\NormalTok{(aves, }\KeywordTok{sqrt}\NormalTok{(aves),}\KeywordTok{log}\NormalTok{(aves))}
\end{Highlighting}
\end{Shaded}

\begin{verbatim}
##    sp1 sp2      sp1      sp2      sp1      sp2
## 1   53 213 7.280110 14.59452 3.970292 5.361292
## 2    1 142 1.000000 11.91638 0.000000 4.955827
## 3   26 114 5.099020 10.67708 3.258097 4.736198
## 4   25 241 5.000000 15.52417 3.218876 5.484797
## 5   70 161 8.366600 12.68858 4.248495 5.081404
## 6   23 166 4.795832 12.88410 3.135494 5.111988
## 7   61 240 7.810250 15.49193 4.110874 5.480639
## 8   76 184 8.717798 13.56466 4.330733 5.214936
## 9   78 237 8.831761 15.39480 4.356709 5.468060
## 10   6 208 2.449490 14.42221 1.791759 5.337538
\end{verbatim}

\begin{Shaded}
\begin{Highlighting}[]
\KeywordTok{cbind}\NormalTok{(insectos, }\KeywordTok{sqrt}\NormalTok{(insectos),}\KeywordTok{log}\NormalTok{(insectos))}
\end{Highlighting}
\end{Shaded}

\begin{verbatim}
##    sp1  sp2      sp1      sp2      sp1      sp2
## 1   72 1851 8.485281 43.02325 4.276666 7.523481
## 2   98 1358 9.899495 36.85105 4.584967 7.213768
## 3   52 2316 7.211103 48.12484 3.951244 7.747597
## 4   50 1980 7.071068 44.49719 3.912023 7.590852
## 5   64 1722 8.000000 41.49699 4.158883 7.451242
## 6   79 2452 8.888194 49.51767 4.369448 7.804659
## 7   47 1687 6.855655 41.07311 3.850148 7.430707
## 8   94 1929 9.695360 43.92038 4.543295 7.564757
## 9   49 1579 7.000000 39.73663 3.891820 7.364547
## 10  96 1009 9.797959 31.76476 4.564348 6.916715
\end{verbatim}

La estandarización de los datos permite modificar las variables
transformándolas en unidades de desviación típica, lo que nos permite
comparar entre valores de distribuciones normales diferentes, o de
valores diferentes.

La estandarización o tipificación se lo realiza restando a cada valor el
valor medio de la variable y dividiendo para la desviación estándar.

\begin{Shaded}
\begin{Highlighting}[]
\NormalTok{avesE <-}\StringTok{ }\NormalTok{(aves[,}\DecValTok{1}\NormalTok{]-}\KeywordTok{mean}\NormalTok{(aves[,}\DecValTok{1}\NormalTok{]))/}\KeywordTok{sd}\NormalTok{(aves[,}\DecValTok{1}\NormalTok{])}
\NormalTok{avesE}
\end{Highlighting}
\end{Shaded}

\begin{verbatim}
##  [1]  0.3819546 -1.4073822 -0.5471241 -0.5815345  0.9669301 -0.6503551
##  [7]  0.6572372  1.1733920  1.2422126 -1.2353306
\end{verbatim}

\begin{Shaded}
\begin{Highlighting}[]
\KeywordTok{round}\NormalTok{(}\KeywordTok{mean}\NormalTok{(avesE),}\DecValTok{1}\NormalTok{);}\KeywordTok{sd}\NormalTok{(avesE) }
\end{Highlighting}
\end{Shaded}

\begin{verbatim}
## [1] 0
\end{verbatim}

\begin{verbatim}
## [1] 1
\end{verbatim}

Como vemos las variables estandarizadas tienen como propiedad que la
desviación estándar es 1 y la media es 0.

\begin{center}\rule{0.5\linewidth}{\linethickness}\end{center}

\chapter{Ejercicio práctico}\label{ejercicio-practico}

\begin{center}\rule{0.5\linewidth}{\linethickness}\end{center}

Una de las preguntas básicas de un ecólogo es saber ¿Cómo de diferentes
son dos comunidades?. Como hemos visto en el capítulo anterior existen
varias decisiones que los investigadores debemos tomar, estas decisiones
afectan directamente a los resultados que podemos obtener y por ende a
las conclusiones biológicas que obtenemos de este análisis.

El presente ejercicio evaluaremos como las diferentes desiciones que
tomamos entorno al procesamiento de datos afectan nuestras medidas de
similitud, y cuales son las conclusiones biológicas que obtenemos con
uno u otro procedimiento. En la tabla \ref{tab:ejer1} mostramos cinco
comunidades hipotéticas.

\begin{table}

\caption{\label{tab:ejer1}Comunidades hipotéticas}
\centering
\begin{tabular}[t]{lrrrrrrrr}
\toprule
  & sp1 & sp2 & sp3 & sp4 & sp5 & sp6 & sp7 & sp8\\
\midrule
A & 26 & 17 & 16 & 1995 & 159 & 0 & 362 & 0\\
B & 0 & 35 & 14 & 236 & 54 & 0 & 496 & 57\\
C & 24 & 0 & 26 & 17 & 88 & 18 & 907 & 20\\
D & 35 & 18 & 24 & 2033 & 175 & 15 & 376 & 16\\
E & 105 & 129 & 40 & 18 & 191 & 53 & 964 & 134\\
\bottomrule
\end{tabular}
\end{table}

Con los datos anteriores:

\begin{enumerate}
\def\labelenumi{\alph{enumi}.}
\item
  Convierta los datos en abundancia relativa por especie (la suma en
  cada especie debe ser igual a 1). Dibuje dos gráficas para
  representar; i) la abundancia total y ii) abundancia relativa de cada
  localidad. ¿Qué diferencias puede ver en la gráfica i y en la ii?¿Qué
  implicaciones biológicas podría tener si utilizamos la primera o la
  segunda matriz para calcular las similitudes?
\item
  Calcule la distancia Euclideana y de Bray Curtis para cada sitio con
  las dos medidas de abundancia y grafíquelas utilizando el NMDS. ¿Cómo
  cambia entre distancias y abundancias? ¿Por qué se dan estas
  diferencias? ¿Puede darle una explicación biológica a los diferentes
  resultados?
\item
  Evalúe la similitud (Sorensen) y el porcentaje de similitud entre
  pares de sitios. ¿Cuáles son los sitios más similares? ¿Cuál es la
  razón de las diferencias entre los índices utilizados? ¿De una
  interpretación biológica a estos resultados?
\end{enumerate}

\chapter*{Prefacio}\label{prefacio-1}
\addcontentsline{toc}{chapter}{Prefacio}

Placeholder

\chapter{Introducción}\label{introduccion-1}

\begin{center}\rule{0.5\linewidth}{\linethickness}\end{center}

\section{Similitud basada en la
distancia}\label{similitud-basada-en-la-distancia-1}

\begin{quote}
``La abundancia se refiere al número de individuos de una especie en una
determinada área''

--- (Smith and Smith 2010)
\end{quote}

Cuando hablamos de la composición de especies de una comunidad nos
referimos al conjunto de especies que habitan una determinada localidad.
Típicamente, esto incluye cierto grado de abundancia de cada especie,
pero puede también ser simplemente un listado de especies en esa
localidad, donde se registra la presencia o ausencia de cada especie.
Ahora, imaginemos que tenemos cuatro localidades (A, B, C, D) donde
recogemos los datos de densidad de dos especies; \emph{Tabebuia
billbergii} y \emph{Geofroea spinosa}, especies características de
bosques secos tropicales. Podemos introducir datos hipotéticos de
abundancia para cada especie en cada una de las localidades.

\begin{Shaded}
\begin{Highlighting}[]
\NormalTok{dens <-}\StringTok{ }\KeywordTok{data.frame}\NormalTok{(}\DataTypeTok{T.bil =} \KeywordTok{c}\NormalTok{(}\DecValTok{1}\NormalTok{, }\DecValTok{1}\NormalTok{, }\DecValTok{2}\NormalTok{, }\DecValTok{3}\NormalTok{), }\DataTypeTok{G.spi =} \KeywordTok{c}\NormalTok{(}\DecValTok{21}\NormalTok{, }\DecValTok{8}\NormalTok{, }\DecValTok{13}\NormalTok{, }\DecValTok{5}\NormalTok{)) }
\KeywordTok{row.names}\NormalTok{(dens) <-}\StringTok{ }\NormalTok{LETTERS[}\DecValTok{1}\NormalTok{:}\DecValTok{4}\NormalTok{]}
\NormalTok{dens}
\end{Highlighting}
\end{Shaded}

\begin{verbatim}
##   T.bil G.spi
## A     1    21
## B     1     8
## C     2    13
## D     3     5
\end{verbatim}

Generamos un gráfico para ver cuánto se parece cada sitio

\begin{Shaded}
\begin{Highlighting}[]
\KeywordTok{par}\NormalTok{(}\DataTypeTok{mar=}\KeywordTok{c}\NormalTok{(}\DecValTok{4}\NormalTok{,}\DecValTok{4}\NormalTok{,}\DecValTok{1}\NormalTok{,}\DecValTok{1}\NormalTok{), }\DataTypeTok{mgp=}\KeywordTok{c}\NormalTok{(}\DecValTok{1}\NormalTok{,}\FloatTok{0.3}\NormalTok{,}\DecValTok{0}\NormalTok{), }\DataTypeTok{tcl=} \NormalTok{-}\FloatTok{0.2}\NormalTok{)}
\KeywordTok{plot}\NormalTok{(dens, }\DataTypeTok{type =} \StringTok{"n"}\NormalTok{, }\DataTypeTok{cex.axis=}\FloatTok{0.8}\NormalTok{, }\DataTypeTok{xlim=}\KeywordTok{c}\NormalTok{(}\DecValTok{0}\NormalTok{,}\DecValTok{20}\NormalTok{)) }
\KeywordTok{text}\NormalTok{(dens, }\KeywordTok{row.names}\NormalTok{(dens), }\DataTypeTok{col =}\StringTok{"blue"}\NormalTok{)}
\end{Highlighting}
\end{Shaded}

\begin{figure}[htbp]
\centering
\includegraphics{Similitud_files/figure-latex/unnamed-chunk-19-1.pdf}
\caption{\label{fig:unnamed-chunk-19}Distancias de cuatro localidades
hipotéticas}
\end{figure}

En la figura \ref{fig:NMDS} vemos que la composición de especies en el
sitio A es diferente de la composición del sitio D. Es decir, la
distancia entre el sitio A y D es mayor que entre los otros sitios. Lo
siguiente que nos deberíamos preguntar es; ¿qué tan distantes están los
dos sitios? Claramente, esto depende de la escala de medición (los
valores de los ejes), y sobre cómo medimos la distancia a través del
espacio multivariado \citep{Stevens2009}.

Estas diferencias entre sitios son dependientes de la abundancia de cada
especie. En el caso de \emph{G. spinosa} su eje varía entre 5 y 21,
mientras que para \emph{T. billbergii} varía entre 1 y 3. Veamos ahora
que sucede con las similitud si incremento la abundancia de \emph{T.
billbergii}.

\begin{Shaded}
\begin{Highlighting}[]
\KeywordTok{par}\NormalTok{(}\DataTypeTok{mar=}\KeywordTok{c}\NormalTok{(}\DecValTok{4}\NormalTok{,}\DecValTok{4}\NormalTok{,}\DecValTok{1}\NormalTok{,}\DecValTok{1}\NormalTok{), }\DataTypeTok{mgp=}\KeywordTok{c}\NormalTok{(}\DecValTok{1}\NormalTok{,}\FloatTok{0.3}\NormalTok{,}\DecValTok{0}\NormalTok{), }\DataTypeTok{tcl=} \NormalTok{-}\FloatTok{0.2}\NormalTok{)}
\NormalTok{dens1 <-}\StringTok{ }\NormalTok{dens}
\NormalTok{dens1$T.bil <-}\StringTok{ }\NormalTok{dens1$T.bil*}\DecValTok{100}
\KeywordTok{plot}\NormalTok{(dens1, }\DataTypeTok{type =} \StringTok{"n"}\NormalTok{, }\DataTypeTok{cex.axis=}\FloatTok{0.8}\NormalTok{, }\DataTypeTok{ylim=}\KeywordTok{c}\NormalTok{(}\DecValTok{0}\NormalTok{,}\DecValTok{300}\NormalTok{)) }
\KeywordTok{text}\NormalTok{(dens1, }\KeywordTok{row.names}\NormalTok{(dens1), }\DataTypeTok{col =}\StringTok{"blue"}\NormalTok{)}
\end{Highlighting}
\end{Shaded}

\begin{figure}[htbp]
\centering
\includegraphics{Similitud_files/figure-latex/2-1.pdf}
\caption{\label{fig:2}Distancias de cuatro localidades hipotéticas}
\end{figure}

Como vemos en la figura \ref{fig:2} las distancias entre cada uno de los
sitios cambio, aunque la comunidad se mantuvo igual. Una forma de
corregir esta distorsión es calcular la densidad relativa de cada
especie, de esta forma cada especie variará entre 0 y 1
\citep{Stevens2009}. Cuando nos referimos a densidad relativa hablamos
de la densidad de una especie con referencia a algo, en relación a la
abundancia de individuos de la misma especie en otros sitios.

Para calcular la densidad relativa dividimos la abundancia de cada
especie para la suma total de los individuos de las especies en esa
muestra.

\begin{Shaded}
\begin{Highlighting}[]
\NormalTok{dens[,}\DecValTok{1}\NormalTok{]/}\KeywordTok{sum}\NormalTok{(dens[,}\DecValTok{1}\NormalTok{])}
\end{Highlighting}
\end{Shaded}

\begin{verbatim}
## [1] 0.1428571 0.1428571 0.2857143 0.4285714
\end{verbatim}

\begin{Shaded}
\begin{Highlighting}[]
\NormalTok{dens1[,}\DecValTok{1}\NormalTok{]/}\KeywordTok{sum}\NormalTok{(dens1[,}\DecValTok{1}\NormalTok{])}
\end{Highlighting}
\end{Shaded}

\begin{verbatim}
## [1] 0.1428571 0.1428571 0.2857143 0.4285714
\end{verbatim}

Ahora podemos ver cómo \emph{T. billbergii} varía en su abundancia en
los cuatro sitios. El sitio A y B tienen el 14\% de individuos mientras
que el D tiene el 42\% de los individuos de esta especie.
Interesantemente, no hay diferencias en las proporciones entre las dos
medidas que tenemos. ¿Qué pasó con las distancias?

\begin{Shaded}
\begin{Highlighting}[]
\NormalTok{dens3 <-}\StringTok{ }\NormalTok{dens}

\NormalTok{for(i in }\DecValTok{1}\NormalTok{:}\DecValTok{2}\NormalTok{)\{}
\NormalTok{dens3[,i] <-}\StringTok{ }\NormalTok{dens[,i]/}\KeywordTok{sum}\NormalTok{(dens[,i])}
\NormalTok{\}}
\NormalTok{dens4 <-}\StringTok{ }\NormalTok{dens1}

\NormalTok{for(i in }\DecValTok{1}\NormalTok{:}\DecValTok{2}\NormalTok{)\{}
\NormalTok{dens4[,i] <-}\StringTok{ }\NormalTok{dens1[,i]/}\KeywordTok{sum}\NormalTok{(dens1[,i])}
\NormalTok{\}}

\KeywordTok{par}\NormalTok{(}\DataTypeTok{mfcol=}\KeywordTok{c}\NormalTok{(}\DecValTok{2}\NormalTok{,}\DecValTok{2}\NormalTok{), }\DataTypeTok{mar=}\KeywordTok{c}\NormalTok{(}\DecValTok{4}\NormalTok{,}\DecValTok{4}\NormalTok{,}\DecValTok{1}\NormalTok{,}\DecValTok{1}\NormalTok{), }\DataTypeTok{mgp=}\KeywordTok{c}\NormalTok{(}\DecValTok{1}\NormalTok{,}\FloatTok{0.3}\NormalTok{,}\DecValTok{0}\NormalTok{), }\DataTypeTok{tcl=} \NormalTok{-}\FloatTok{0.2}\NormalTok{)}
\KeywordTok{plot}\NormalTok{(dens, }\DataTypeTok{type =} \StringTok{"n"}\NormalTok{, }\DataTypeTok{cex.axis=}\FloatTok{0.8}\NormalTok{, }\DataTypeTok{xlim=}\KeywordTok{c}\NormalTok{(}\DecValTok{0}\NormalTok{,}\DecValTok{20}\NormalTok{), }\DataTypeTok{main =} \StringTok{"Densidad"}\NormalTok{) }
\KeywordTok{text}\NormalTok{(dens, }\KeywordTok{row.names}\NormalTok{(dens), }\DataTypeTok{col =}\StringTok{"blue"}\NormalTok{)}

\KeywordTok{plot}\NormalTok{(dens1, }\DataTypeTok{type =} \StringTok{"n"}\NormalTok{, }\DataTypeTok{cex.axis=}\FloatTok{0.8}\NormalTok{, }\DataTypeTok{ylim=}\KeywordTok{c}\NormalTok{(}\DecValTok{0}\NormalTok{,}\DecValTok{300}\NormalTok{), }\DataTypeTok{main =} \StringTok{"Densidad 2"}\NormalTok{) }
\KeywordTok{text}\NormalTok{(dens1, }\KeywordTok{row.names}\NormalTok{(dens1), }\DataTypeTok{col =}\StringTok{"blue"}\NormalTok{)}

\KeywordTok{plot}\NormalTok{(dens3, }\DataTypeTok{type =} \StringTok{"n"}\NormalTok{, }\DataTypeTok{cex.axis=}\FloatTok{0.8}\NormalTok{, }\DataTypeTok{main =} \StringTok{"Densidad relativa"}\NormalTok{) }
\KeywordTok{text}\NormalTok{(dens3, }\KeywordTok{row.names}\NormalTok{(dens3), }\DataTypeTok{col =}\StringTok{"blue"}\NormalTok{)}

\KeywordTok{plot}\NormalTok{(dens4, }\DataTypeTok{type =} \StringTok{"n"}\NormalTok{, }\DataTypeTok{cex.axis=}\FloatTok{0.8}\NormalTok{, }\DataTypeTok{main =} \StringTok{"Densidad relativa2"}\NormalTok{) }
\KeywordTok{text}\NormalTok{(dens4, }\KeywordTok{row.names}\NormalTok{(dens4), }\DataTypeTok{col =}\StringTok{"blue"}\NormalTok{)}
\end{Highlighting}
\end{Shaded}

\begin{figure}[htbp]
\centering
\includegraphics{Similitud_files/figure-latex/3-1.pdf}
\caption{\label{fig:3}Distancias de cuatro localidades hipotéticas}
\end{figure}

En la figura \ref{fig:3} podemos apreciar que no hay diferencias entre
las dos densidades cuando estoy usando la densidad relativa. Pero ¿Qué
implicaciones biológicas tiene el usar las densidades relativas para
calcular la distancia entre sitios?

Cuando usamos las densidades relativas lo que estamos haciendo es darles
el mismo peso a tods las especies, de esta manera si yo tengo un
ecosistema con una especie dominante y varias subordinadas, al usar la
densidad relativa estoy eliminando esa dominancia. Es importante que
tener claro este punto ya que las interpretaciones que puedo hacer con
los datos de densidad y densidad relativa son distintos.

Ya sea que nuestras medidas de abundancia son absoluta o relativa, nos
interesa conocer cuan diferente es la comunidad de una muestra (o sitio)
con relación a la otra. En el ejemplo ha sido fácil entender la
diferencia entre las dos comunidades debido a que teníamos únicamente
dos especies, pero con más de tres especies es complicado observar estas
diferencias gráficamente. Tal vez la forma más sencilla de describir la
diferencia entre los sitios es calcular las \emph{distancias} entre cada
par de sitios.

\subsection{Distancias entre sitios}\label{distancias-entre-sitios-1}

La \emph{distancia} entre dos muestras está dada por la diferencia entre
la abundancia y la composición de especies, como lo hemos visto esto
genera una distancia, en el caso del ejemplo la comunidad A esta más
alejada de la comunidad D que de las otras dos.

Existen muchas formas de poder calcular las distancias entre estos
puntos una de las más sencillas es la distancia \emph{Euclidiana}. La
distancia euclidiana entre dos sitios es simplemente la longitud del
vector que conecta los sitios y la podemos obtener como
\(\sqrt{x^2+y^2}\), donde \emph{``x''} y \emph{``y''} son las
coordenadas (x, y) de distancia entre un par de sitios.

En nuestro caso si queremos comparar B y C tenemos que la distancia en
el eje \emph{x} es la diferencia de la abundancia de \emph{T. bilbergii}
entre el sitio B y C.

\begin{Shaded}
\begin{Highlighting}[]
\NormalTok{x <-}\StringTok{ }\NormalTok{dens[}\DecValTok{2}\NormalTok{, }\DecValTok{1}\NormalTok{] -}\StringTok{ }\NormalTok{dens[}\DecValTok{3}\NormalTok{, }\DecValTok{1}\NormalTok{]}
\end{Highlighting}
\end{Shaded}

Mientras que la distancia en el eje \emph{y} es la diferencia en la
abundancia de \emph{G. spinosa} entre el sitio B y C.

\begin{Shaded}
\begin{Highlighting}[]
\NormalTok{y <-}\StringTok{ }\NormalTok{dens[}\DecValTok{2}\NormalTok{, }\DecValTok{2}\NormalTok{] -}\StringTok{ }\NormalTok{dens[}\DecValTok{3}\NormalTok{, }\DecValTok{2}\NormalTok{]}
\end{Highlighting}
\end{Shaded}

Ahora obtenemos las distancias entre los dos sitios

\begin{Shaded}
\begin{Highlighting}[]
\KeywordTok{sqrt}\NormalTok{(x^}\DecValTok{2} \NormalTok{+}\StringTok{ }\NormalTok{y^}\DecValTok{2}\NormalTok{)}
\end{Highlighting}
\end{Shaded}

\begin{verbatim}
## [1] 5.09902
\end{verbatim}

Pero como en \emph{R} todo es sencillo podemos utilizar la función
\emph{dist}

\begin{Shaded}
\begin{Highlighting}[]
\KeywordTok{dist}\NormalTok{(dens)}
\end{Highlighting}
\end{Shaded}

\begin{verbatim}
##           A         B         C
## B 13.000000                    
## C  8.062258  5.099020          
## D 16.124515  3.605551  8.062258
\end{verbatim}

Si bien este cálculo es sencillo con dos especies, si tenemos que
calcular la distancia para una comunidad con más de tres especies los
cálculos son tediosos y largos. Para calcular la distancia
\emph{Euclidiana} entre pares de sitios con \emph{R} especies utilizamos
la siguiente ecuación:

\begin{quote}
\[D_E = \sqrt{\sum_{i=l}^R (x_{ai} - x_{bi})^2}\] Distancia Euclidiana
\end{quote}

Existen otras formas de medir distancias entre dos localidades. En
ecología una de las distancias más utilizada es la distancia de
\emph{Bray-Curtis}, conocida también como \emph{Sorensen}. Esta
distancia es calculada como:

\begin{quote}
\[D_{BC} = \sum_{i=l}^R \frac{(x_{ai} - x_{bi})}{(x_{ai} + x_{bi})}\]
Distancia de Bray-Curtis
\end{quote}

La distancia \emph{Bray-Curtis} no es más que la diferencia total en la
abundancia de especies entre dos sitios, dividido para la abundancia
total en cada sitio. La distancia Bray-Curtis tiende a resultar más
intuitiva debido a que las especies comunes y raras tienen pesos
relativamente similares, mientras que la distancia euclidia depende en
mayor medida de las especies más abundantes. Esto sucede porque las
distancias euclidianas se basan en diferencias al cuadrado, mientras que
Bray-Curtis utiliza diferencias absolutas. El elevar un número al
cuadrado siempre amplifica la importancia de los valores más grandes. En
la figura \ref{fig:bray} se compara gráficos basados en distancias
euclidianas y Bray-Curtis de los mismos datos.

Como se había comentado es virtualmente imposible representar una
distancia en más de tres dimensiones (cada especie es una dimensión).
Una forma sencilla de mostrar distancias para tres o más especies es
crear un gráfico de dos dimensiones, intentando organizar todos los
sitios para que las distancias sean aproximadamente las correctas. Está
claro que esto es una aproximación nunca estas serán exactas. Una
técnica que intenta crear un arreglo aproximado es escalamiento
multidimensional no métrico (NMDS). Vamos a calcular las distancias para
nuestra comunidad, primero vamos a añadir dos especies más a nuestra
comunidad, \emph{Ceiba trichistandra} y \emph{Colicodendron scabridum}.

\begin{Shaded}
\begin{Highlighting}[]
\NormalTok{dens$C.tri<-}\StringTok{ }\KeywordTok{c}\NormalTok{(}\DecValTok{11}\NormalTok{, }\DecValTok{3}\NormalTok{, }\DecValTok{7}\NormalTok{, }\DecValTok{5}\NormalTok{)}
\NormalTok{dens$C.sca<-}\StringTok{ }\KeywordTok{c}\NormalTok{(}\DecValTok{16}\NormalTok{, }\DecValTok{0}\NormalTok{, }\DecValTok{9}\NormalTok{, }\DecValTok{4}\NormalTok{)}
\end{Highlighting}
\end{Shaded}

La función de escalamiento multidimensional no-métrico está en el
paquete \texttt{vegan}. Aquí mostramos las distancias euclidianas entre
sitios (Figura \ref{fig:bray}a) y las distancias de Bray-Curtis (Figura
\ref{fig:bray}b).

\begin{Shaded}
\begin{Highlighting}[]
\KeywordTok{library}\NormalTok{(vegan) }

\CommentTok{#Distancia Euclidiana}
\NormalTok{mdsE <-}\StringTok{ }\KeywordTok{metaMDS}\NormalTok{(dens, }\DataTypeTok{distance =} \StringTok{"euc"}\NormalTok{, }\DataTypeTok{autotransform =} \OtherTok{FALSE}\NormalTok{, }\DataTypeTok{trace =} \DecValTok{0}\NormalTok{) }
\CommentTok{#Distancia de Bray-Curtis}
\NormalTok{mdsB <-}\StringTok{ }\KeywordTok{metaMDS}\NormalTok{(dens, }\DataTypeTok{distance =} \StringTok{"bray"}\NormalTok{, }\DataTypeTok{autotransform =} \OtherTok{FALSE}\NormalTok{, }\DataTypeTok{trace =} \DecValTok{0}\NormalTok{) }
\end{Highlighting}
\end{Shaded}

\begin{Shaded}
\begin{Highlighting}[]
\KeywordTok{par}\NormalTok{(}\DataTypeTok{mfcol=}\KeywordTok{c}\NormalTok{(}\DecValTok{1}\NormalTok{,}\DecValTok{2}\NormalTok{), }\DataTypeTok{oma=}\KeywordTok{c}\NormalTok{(}\DecValTok{1}\NormalTok{,}\DecValTok{1}\NormalTok{,}\DecValTok{1}\NormalTok{,}\DecValTok{1}\NormalTok{), }\DataTypeTok{mar=}\KeywordTok{c}\NormalTok{(}\DecValTok{4}\NormalTok{,}\DecValTok{4}\NormalTok{,}\DecValTok{1}\NormalTok{,}\DecValTok{1}\NormalTok{),}
    \DataTypeTok{mgp=}\KeywordTok{c}\NormalTok{(}\DecValTok{1}\NormalTok{,}\FloatTok{0.3}\NormalTok{,}\DecValTok{0}\NormalTok{), }\DataTypeTok{tcl=} \NormalTok{-}\FloatTok{0.2}\NormalTok{)}

\KeywordTok{plot}\NormalTok{(mdsE, }\DataTypeTok{display =} \StringTok{"sites"}\NormalTok{, }
     \DataTypeTok{type =} \StringTok{"text"}\NormalTok{,}\DataTypeTok{main=}\StringTok{"a)Euclidiana"}\NormalTok{, }
     \DataTypeTok{cex.axis=} \FloatTok{0.7}\NormalTok{, }\DataTypeTok{cex.main=}\FloatTok{0.75}\NormalTok{, }\DataTypeTok{cex.lab=}\FloatTok{0.7}\NormalTok{)}

\KeywordTok{plot}\NormalTok{(mdsB, }\DataTypeTok{display =} \StringTok{"sites"}\NormalTok{, }\DataTypeTok{type =} \StringTok{"text"}\NormalTok{, }
     \DataTypeTok{main=}\StringTok{"b)Bray-Curtis"}\NormalTok{, }
     \DataTypeTok{cex.axis=} \FloatTok{0.7}\NormalTok{, }\DataTypeTok{cex.main=}\FloatTok{0.75}\NormalTok{, }\DataTypeTok{cex.lab=}\FloatTok{0.7}\NormalTok{)}
\end{Highlighting}
\end{Shaded}

\begin{figure}[htbp]
\centering
\includegraphics{Similitud_files/figure-latex/unnamed-chunk-27-1.pdf}
\caption{\label{fig:unnamed-chunk-27}Arreglo de las parcelas en distancias
multidimensionales no métricas (NMDS). Estas dos figuras muestran los
mismos datos en bruto, pero las distancias euclidianas tienden a
enfatizar las diferencias debidas a las especies más abundantes,
mientras que Bray-Curtis no lo hace.}
\end{figure}

\section{Similitud}\label{similitud-1}

Ahora que sabemos cuan distantes son los diferentes sitios, muchas veces
nos podría interesar cuan similares son cada uno de los sitios a
continuación se describen dos medidas de similitud; \emph{Porcentaje de
Similitud} e \emph{Índice de Sorensen}.

El \emph{porcentaje de similitud} puede ser simplemente la suma de los
porcentajes mínimos de cada especie en la comunidad. Lo primero que
debemos hacer es convertir la abundancia de cada especie a su abundancia
relativa dentro de cada sitio. Para ello dividimos la abundancia de cada
especie por la suma de las abundancias en cada sitio.

\begin{Shaded}
\begin{Highlighting}[]
\NormalTok{dens.RA <-}\StringTok{ }\KeywordTok{t}\NormalTok{(}\KeywordTok{apply}\NormalTok{(dens, }\DecValTok{1}\NormalTok{, function(sp.abun) sp.abun/}\KeywordTok{sum}\NormalTok{(sp.abun)))}
\NormalTok{dens.RA}
\end{Highlighting}
\end{Shaded}

\begin{verbatim}
##        T.bil     G.spi     C.tri     C.sca
## A 0.02040816 0.4285714 0.2244898 0.3265306
## B 0.08333333 0.6666667 0.2500000 0.0000000
## C 0.06451613 0.4193548 0.2258065 0.2903226
## D 0.17647059 0.2941176 0.2941176 0.2352941
\end{verbatim}

El siguiente paso para comparar entre sitios, es encontrar el valor
mínimo para cada especie entre los sitios que debemos comparar. Vamos a
comparar los sitios A y B, para esto utilizamos la función
\texttt{aplly}, la cual nos permite encontrar el valor mínimo entre las
filas 1 y 2 (sitio A y B respectivamente). Para \emph{T. billbergi} en
el sitio A la abundancia relativa es 0.02 que es menor a la abundancia
en el sitio B que es de 0.08.

\begin{Shaded}
\begin{Highlighting}[]
\NormalTok{mins <-}\StringTok{ }\KeywordTok{apply}\NormalTok{(dens.RA[}\DecValTok{1}\NormalTok{:}\DecValTok{2}\NormalTok{, ], }\DecValTok{2}\NormalTok{, min)}
\NormalTok{mins}
\end{Highlighting}
\end{Shaded}

\begin{verbatim}
##      T.bil      G.spi      C.tri      C.sca 
## 0.02040816 0.42857143 0.22448980 0.00000000
\end{verbatim}

Finalmente para conocer el porcentaje de similitud entre los dos sitios
sumamos estos valores y multiplicamos por 100.

\begin{Shaded}
\begin{Highlighting}[]
\KeywordTok{sum}\NormalTok{(mins)*}\DecValTok{100}
\end{Highlighting}
\end{Shaded}

\begin{verbatim}
## [1] 67.34694
\end{verbatim}

Esto significa que la comunidad A y B tienen un porcentaje de similitud
del 67\%.

El índice de Sorensen es la segunda medida de similitud que vamos a
estudiar, este índice es medido como:

\begin{quote}
\[S_s= \frac{(2C)}{(A+B)}\] Índice de Sorensen
\end{quote}

Donde \emph{C} es el número de especies en común entre los dos sitios, y
\emph{A} y \emph{B} son el número de especies en cada sitio. Esto es
equivalente a dividir las especies compartidas por la riqueza media.

Para calcular el índice de Sorensen entre los sitios A y B necesitamos
definir el número de especies compartidas y luego la riqueza de cada uno
de los dos sitios.

Definimos si alguna de las especies en uno de los sitios la abundancia
no es igual a cero, eso nos dirá en qué casos se comparten especies.
Finalmente, sumamos todas las especies que su abundancia es mayor a
cero.

\begin{Shaded}
\begin{Highlighting}[]
\NormalTok{comp<-}\StringTok{ }\KeywordTok{apply}\NormalTok{(dens[}\DecValTok{1}\NormalTok{:}\DecValTok{2}\NormalTok{, ], }\DecValTok{2}\NormalTok{, function(abuns) }\KeywordTok{all}\NormalTok{(abuns !=}\StringTok{ }\DecValTok{0}\NormalTok{))}
\NormalTok{comp}
\end{Highlighting}
\end{Shaded}

\begin{verbatim}
## T.bil G.spi C.tri C.sca 
##  TRUE  TRUE  TRUE FALSE
\end{verbatim}

\begin{Shaded}
\begin{Highlighting}[]
\NormalTok{Rs <-}\StringTok{ }\KeywordTok{apply}\NormalTok{(dens[}\DecValTok{1}\NormalTok{:}\DecValTok{2}\NormalTok{, ], }\DecValTok{1}\NormalTok{, function(x) }\KeywordTok{sum}\NormalTok{(x >}\StringTok{ }\DecValTok{0}\NormalTok{))}
\NormalTok{Rs}
\end{Highlighting}
\end{Shaded}

\begin{verbatim}
## A B 
## 4 3
\end{verbatim}

Como vemos, la abundancia de \emph{C. scabridum} en uno de los dos
sitios es igual a Cero, lo confirmamos al tener la riqueza por sitio. El
sitio B tenemos únicamente 3 especies.

Ahora aplicamos la formula, dividimos las especies compartidas
(\emph{comp}) para la riqueza total de los dos sitios y lo multiplicamos
por 2.

\begin{Shaded}
\begin{Highlighting}[]
\NormalTok{(}\DecValTok{2}\NormalTok{*}\KeywordTok{sum}\NormalTok{(comp))/}\KeywordTok{sum}\NormalTok{(Rs)}
\end{Highlighting}
\end{Shaded}

\begin{verbatim}
## [1] 0.8571429
\end{verbatim}

Según el índice de Sorensen estos dos sitios son parecidos en un 86\%.
Los datos de los dos índices utilizados difieren entre sí, el porcentaje
de similitud utiliza no solamente la presencia ausencia sino también la
abundancia lo que podría estar reduciendo la similitud entre sitios.

\begin{FOO}
La distancia de Bray Curtis y el índice de Sorensen tienen una base
conceptual similar, la diferencia es que Bray-Curtis se basa en datos de
abundancias y Sorensen en datos de presencia ausencia. Muchas veces se
utilizan como sinónimos aunque se especifíca si esta basado en
abundancias o en presencia-ausencia.
\end{FOO}

\section{Transformación y Estandarización de
datos}\label{transformacion-y-estandarizacion-de-datos-1}

Cuando trabajamos con datos multivariantes cabe la posibilidad de que
los datos dentro de esta matriz tengan diferencias de medidas
importantes. Como vimos antes el cálculo de distancia entre los sitios
puede verse fuertemente afectado por los datos y la magnitud de sus
diferencias.

\begin{Shaded}
\begin{Highlighting}[]
\KeywordTok{library}\NormalTok{(readxl)}
\KeywordTok{library}\NormalTok{(knitr)}
\NormalTok{dta <-}\StringTok{ }\KeywordTok{read_excel}\NormalTok{(}\StringTok{"bentos.xlsx"}\NormalTok{)}
\KeywordTok{kable}\NormalTok{(dta)}
\end{Highlighting}
\end{Shaded}

\begin{tabular}{l|r|r|r|r|r|r|r|r|r|r|r}
\hline
LOCALIDAD & Alluaudomyia & Atopsyche & Atrichopogon & Baetis & Bezzia & Blepharicera & Ceratopogonidae & Chelifera & Chimarra & Chironominae mfe1 & Colembola mf1\\
\hline
Bo-1 & 0 & 0 & 0 & 6 & 0 & 1 & 0 & 1 & 3 & 18 & 4\\
\hline
Bo-2 & 0 & 0 & 0 & 3 & 0 & 0 & 0 & 0 & 1 & 9 & 0\\
\hline
Bo-3 & 0 & 0 & 0 & 6 & 0 & 0 & 1 & 1 & 1 & 9 & 0\\
\hline
BP-1 & 0 & 3 & 0 & 81 & 0 & 0 & 0 & 0 & 0 & 27 & 0\\
\hline
BP-2 & 0 & 0 & 0 & 9 & 0 & 0 & 0 & 0 & 2 & 0 & 0\\
\hline
BP-3 & 0 & 0 & 0 & 54 & 0 & 0 & 1 & 0 & 0 & 9 & 0\\
\hline
Pa-1 & 1 & 0 & 0 & 984 & 0 & 0 & 0 & 0 & 0 & 81 & 0\\
\hline
Pa-2 & 0 & 0 & 0 & 15 & 0 & 0 & 0 & 0 & 1 & 9 & 0\\
\hline
Pa-3 & 0 & 0 & 0 & 93 & 1 & 0 & 0 & 0 & 0 & 18 & 0\\
\hline
Ur-1 & 0 & 0 & 0 & 6 & 0 & 0 & 0 & 0 & 0 & 855 & 0\\
\hline
Ur-2 & 0 & 0 & 1 & 12 & 0 & 0 & 0 & 1 & 0 & 9 & 0\\
\hline
Ur-3 & 0 & 0 & 0 & 0 & 10 & 0 & 0 & 0 & 0 & 27 & 0\\
\hline
\end{tabular}

La transformación

Imaginemos que estamos trabajando con la comunidad de herbáceas y
existan especies con abundancias máximas de 500, mientras otras especies
no alcanzan 10 individuos, o en comunidades de insectos donde ciertas
especies puedan tener valores de miles y otros de decenas. Si no
transformamos la ordenación estará determinada básicamente por esta
especie.

Al transformar los datos evitamos que las especies más comunes dominen
en el resultado final de la ordenación y aumentamos la influencia de las
especies subordinadas en el modelo resultante.

Existen varias posibilidades para transformar los datos, por lo que
definir que función utilizar para transformar los datos es importante.
Cada tipo de transformación produce resultados distintos por lo que
debemos utilizarlas con precaución.

Las transformaciones más sencilla o menos intensa es la raíz cuadrada,
mientras que el logaritmo es la transformación más intensa, podríamos
utilizar la raíz cuarta como una función intermedia. La raíz cuadrada la
utilizaríamos cuando tenemos diferencias como en el caso de las aves con
variaciones de una magnitud de diferencia (entre decenas y centenas),
mientras que la transformación logarítmica la haríamos con la comunidad
de insectos donde hay más de una magnitud de diferencia (entre decenas y
miles).

Aunque hay muchos autores que aconsejan realizar transformaciones hay
que ser conscientes de lo que estamos haciendo, transformaciones muy
fuertes en una matriz con pocas diferencias pueden hacer que, por
ejemplo, las especies raras tengan igual peso que las dominantes, esto
es lo que queremos?

Veamos un ejemplo:

\begin{Shaded}
\begin{Highlighting}[]
\KeywordTok{set.seed}\NormalTok{(}\DecValTok{4}\NormalTok{)}
\NormalTok{aves<-}\StringTok{ }\KeywordTok{data.frame}\NormalTok{(}\DataTypeTok{sp1=} \KeywordTok{sample}\NormalTok{(}\DecValTok{1}\NormalTok{:}\DecValTok{90}\NormalTok{, }\DecValTok{10}\NormalTok{), }\DataTypeTok{sp2=} \KeywordTok{sample}\NormalTok{(}\DecValTok{100}\NormalTok{:}\DecValTok{250}\NormalTok{, }\DecValTok{10}\NormalTok{))}

\NormalTok{insectos<-}\StringTok{ }\KeywordTok{data.frame}\NormalTok{(}\DataTypeTok{sp1=} \KeywordTok{sample}\NormalTok{(}\DecValTok{5}\NormalTok{:}\DecValTok{99}\NormalTok{, }\DecValTok{10}\NormalTok{), }\DataTypeTok{sp2=} \KeywordTok{sample}\NormalTok{(}\DecValTok{1000}\NormalTok{:}\DecValTok{2500}\NormalTok{, }\DecValTok{10}\NormalTok{))}

\NormalTok{##¿Qué pasa cuando transformamos?}
\KeywordTok{cbind}\NormalTok{(aves, }\KeywordTok{sqrt}\NormalTok{(aves),}\KeywordTok{log}\NormalTok{(aves))}
\end{Highlighting}
\end{Shaded}

\begin{verbatim}
##    sp1 sp2      sp1      sp2      sp1      sp2
## 1   53 213 7.280110 14.59452 3.970292 5.361292
## 2    1 142 1.000000 11.91638 0.000000 4.955827
## 3   26 114 5.099020 10.67708 3.258097 4.736198
## 4   25 241 5.000000 15.52417 3.218876 5.484797
## 5   70 161 8.366600 12.68858 4.248495 5.081404
## 6   23 166 4.795832 12.88410 3.135494 5.111988
## 7   61 240 7.810250 15.49193 4.110874 5.480639
## 8   76 184 8.717798 13.56466 4.330733 5.214936
## 9   78 237 8.831761 15.39480 4.356709 5.468060
## 10   6 208 2.449490 14.42221 1.791759 5.337538
\end{verbatim}

\begin{Shaded}
\begin{Highlighting}[]
\KeywordTok{cbind}\NormalTok{(insectos, }\KeywordTok{sqrt}\NormalTok{(insectos),}\KeywordTok{log}\NormalTok{(insectos))}
\end{Highlighting}
\end{Shaded}

\begin{verbatim}
##    sp1  sp2      sp1      sp2      sp1      sp2
## 1   72 1851 8.485281 43.02325 4.276666 7.523481
## 2   98 1358 9.899495 36.85105 4.584967 7.213768
## 3   52 2316 7.211103 48.12484 3.951244 7.747597
## 4   50 1980 7.071068 44.49719 3.912023 7.590852
## 5   64 1722 8.000000 41.49699 4.158883 7.451242
## 6   79 2452 8.888194 49.51767 4.369448 7.804659
## 7   47 1687 6.855655 41.07311 3.850148 7.430707
## 8   94 1929 9.695360 43.92038 4.543295 7.564757
## 9   49 1579 7.000000 39.73663 3.891820 7.364547
## 10  96 1009 9.797959 31.76476 4.564348 6.916715
\end{verbatim}

La estandarización de los datos permite modificar las variables
transformándolas en unidades de desviación típica, lo que nos permite
comparar entre valores de distribuciones normales diferentes, o de
valores diferentes.

La estandarización o tipificación se lo realiza restando a cada valor el
valor medio de la variable y dividiendo para la desviación estándar.

\begin{Shaded}
\begin{Highlighting}[]
\NormalTok{avesE <-}\StringTok{ }\NormalTok{(aves[,}\DecValTok{1}\NormalTok{]-}\KeywordTok{mean}\NormalTok{(aves[,}\DecValTok{1}\NormalTok{]))/}\KeywordTok{sd}\NormalTok{(aves[,}\DecValTok{1}\NormalTok{])}
\NormalTok{avesE}
\end{Highlighting}
\end{Shaded}

\begin{verbatim}
##  [1]  0.3819546 -1.4073822 -0.5471241 -0.5815345  0.9669301 -0.6503551
##  [7]  0.6572372  1.1733920  1.2422126 -1.2353306
\end{verbatim}

\begin{Shaded}
\begin{Highlighting}[]
\KeywordTok{round}\NormalTok{(}\KeywordTok{mean}\NormalTok{(avesE),}\DecValTok{1}\NormalTok{);}\KeywordTok{sd}\NormalTok{(avesE) }
\end{Highlighting}
\end{Shaded}

\begin{verbatim}
## [1] 0
\end{verbatim}

\begin{verbatim}
## [1] 1
\end{verbatim}

Como vemos las variables estandarizadas tienen como propiedad que la
desviación estándar es 1 y la media es 0.

\begin{center}\rule{0.5\linewidth}{\linethickness}\end{center}

\chapter{Ejercicio práctico}\label{ejercicio-practico-1}

\begin{center}\rule{0.5\linewidth}{\linethickness}\end{center}

Una de las preguntas básicas de un ecólogo es saber ¿Cómo de diferentes
son dos comunidades?. Como hemos visto en el capítulo anterior existen
varias decisiones que los investigadores debemos tomar, estas decisiones
afectan directamente a los resultados que podemos obtener y por ende a
las conclusiones biológicas que obtenemos de este análisis.

El presente ejercicio evaluaremos como las diferentes desiciones que
tomamos entorno al procesamiento de datos afectan nuestras medidas de
similitud, y cuales son las conclusiones biológicas que obtenemos con
uno u otro procedimiento. En la tabla \ref{tab:ejer1} mostramos cinco
comunidades hipotéticas.

\begin{table}

\caption{\label{tab:unnamed-chunk-37}Comunidades hipotéticas}
\centering
\begin{tabular}[t]{lrrrrrrrr}
\toprule
  & sp1 & sp2 & sp3 & sp4 & sp5 & sp6 & sp7 & sp8\\
\midrule
A & 26 & 17 & 16 & 1995 & 159 & 0 & 362 & 0\\
B & 0 & 35 & 14 & 236 & 54 & 0 & 496 & 57\\
C & 24 & 0 & 26 & 17 & 88 & 18 & 907 & 20\\
D & 35 & 18 & 24 & 2033 & 175 & 15 & 376 & 16\\
E & 105 & 129 & 40 & 18 & 191 & 53 & 964 & 134\\
\bottomrule
\end{tabular}
\end{table}

Con los datos anteriores:

\begin{enumerate}
\def\labelenumi{\alph{enumi}.}
\item
  Convierta los datos en abundancia relativa por especie (la suma en
  cada especie debe ser igual a 1). Dibuje dos gráficas para
  representar; i) la abundancia total y ii) abundancia relativa de cada
  localidad. ¿Qué diferencias puede ver en la gráfica i y en la ii?¿Qué
  implicaciones biológicas podría tener si utilizamos la primera o la
  segunda matriz para calcular las similitudes?
\item
  Calcule la distancia Euclideana y de Bray Curtis para cada sitio con
  las dos medidas de abundancia y grafíquelas utilizando el NMDS. ¿Cómo
  cambia entre distancias y abundancias? ¿Por qué se dan estas
  diferencias? ¿Puede darle una explicación biológica a los diferentes
  resultados?
\item
  Evalúe la similitud (Sorensen) y el porcentaje de similitud entre
  pares de sitios. ¿Cuáles son los sitios más similares? ¿Cuál es la
  razón de las diferencias entre los índices utilizados? ¿De una
  interpretación biológica a estos resultados?
\end{enumerate}

\bibliography{packages.bib}


\end{document}
